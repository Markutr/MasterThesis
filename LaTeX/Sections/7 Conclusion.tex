\section{Closing remarks}
\label{section:application2:conclusion}
In this thesis, I have investigated survey data on chronic back pain in United States adults aged 25 to 84 in the period 1997 to 2018, stratified into four educational groups. The goal was to investigate how disparities in back pain across educational groups has evolved using three time scales, namely with age, periods, and birth cohorts. To this end, I implemented a Bayesian multivariate age-period-cohort model using smoothing priors on the latent field. For model inference I used \texttt{INLA} in \texttt{R}. I achieved in smooth incorporation of provided sociological expert knowledge into the hyperpriors on the latent field using the HD and PC prior frameworks through \texttt{makemyprior}. Furthermore, I conducted model selection and prior sensitivity analysis to ensure a robust fit. In addition, I have been working with the complex survey design of the data, and succeeded in accounting for such design in the models using a recently-proposed two-step approach. Lastly, after personal communication with the sociologist who provided the expert knowledge, I investigated how the educational differences in back pain evolved for the white, US-born population only, mitigating potential confounding factors.

By my analysis, I identified interesting educational disparities in back pain over age and birth cohorts. For all educational groups compared to the group of bachelor and above, the educational disparities in back pain are increasing over both age and with more recent birth cohorts. In particular, the group with the lowest educational attainment experienced the largest disparities by far. By mitigating potential confounding factors foreign to the United States population, I found that these disparities are even larger only for the group with the lowest educational attainment compared to the highest. Overall, my results clearly indicate the presence of educational disparities. These results will aid researchers in the field on chronic back pain to make new interpretations and conclusions on the evolution of the educational differences in back pain. 

Additionally, I succeeded in accounting for the survey design using the recently-proposed two-step approach. In this particular application, accounting for the complex survey design did not change the overall interpretation of the trends. Despite this, seeing as the survey design always should be accounted for in the analyses of survey data, I recommend using the procedure used here in future applications. 

%Further things to do
To continue the work on the topics of this thesis, it would be interesting to investigate models using the survey data on an individual-specific format. By doing this, we could account for covariates that are only available on an individual level, such as race and immigration status. The complex survey design cannot be accounted for when using the data on an individual-specific format. I have made strides towards the individual-specific analysis by successfully fitting a model without covariates using data on this format. Therefore, including covariates to this model should prove straightforward. Moreover, to better interpret and understand causes of the observed educational disparities, further collaboration with sociologists is required.

Results of the thesis will be presented by Andrea Riebler in the invited session “Bayesian Statistical Demography” of the 2024 ISBA world meeting in Venice Italy, for more information see \url{https://www.unive.it/web/en/2208/home}.
