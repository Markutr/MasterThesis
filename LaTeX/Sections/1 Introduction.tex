\section{Introduction}
\label{section:Introduction}
%Chronic pain, what is it, how common, how does it affect people
Chronic pain is one of the most common health conditions plaguing the general population, compromising the general and psychological health of the affected individuals \citep{becker1997pain, gureje1998persistent}, and in turn, influencing their social and economic well-being due to their reduced productive functioning \citep{latham1994socioeconomic}. Chronic pain is commonly defined as pain that persists past the healing phase following an injury \citep{merskey1986classification}. These conditions are typically labelled by site (e.g., back, head, viscera) and type (e.g., neuropathic, arthritic, cancer, myofascial, diabetic) \citep{APKARIAN200981} of injury. In personal and work life, chronic pain entails functional impairment leading to the inability to work and provide familial care. Coupled with the emotional distress that this condition may inflict, it is a key factor in quality of life and an enormous personal and economic burden \citep{duenas2016review}. The condition is estimated to have a high annual economic cost for society as a whole \citep{bp-cost}, with reported incidence rates in the United States up to $30\%$ \citep{johannes2010prevalence}. Despite the large societal burden of the condition, chronic pain has long gone unrecognized by international public health institutions, with the World Health Organization (WHO) finally representing the condition systematically in the 11th rendition International Classification of Diseases (ICD-11), which replaced the 10th rendition in 2022 \citep{ICD-11}. Considering the heavy burden caused by chronic pain, sociologists have consistently pushed for the condition to be recognized, all the while carrying out extensive research on the topic in recent decades \citep{gatchel2007biopsychosocial}. 

%Disparities in health by education, how and why
At the same time, in spite of extensive investigation on the biomedical, psychological and sociological aspects of chronic pain, questions regarding the trends in chronic pain remain unanswered. Among these questions lie the question of how trends in chronic pain have evolved over time, and how socioeconomic factors may have influenced these trends \citep{Zajacova2021-ee}. Education is one such factor, which has been investigated in several studies and is suggested as a powerful determinant of health \citep{mirowsky2017education}. Research has shown strong stratification in the health outcomes by level of attained education, with a lower prevalence of chronic conditions among individuals with higher levels of attained education \citep{dionne2001formal,kennedy2014prevalence}. In particular, recipients of the general educational development (GED) diploma, intended to be equivalent to a high school (HS) diploma, suffers from disproportionately worse health outcomes compared to individuals with higher attained education, such as bachelor's or master's degrees \citep{ZAJACOVA20201270}. The positive effect of education on health has been observed not only for chronic conditions, but also for other health conditions, such as functional limitations, disabilities, depressive symptoms, and active life expectancy \citep{montez2012educational,lorant2003socioeconomic,schoeni2005persistent}. Chronic pain also has the potential to act as a barometer for general health and well-being in a population, further underscoring the importance of research on the temporal trends in chronic pain with stratification by education \citep{sociology}.

%NHIS data and complex survey design
Research in the field is typically carried out using survey data collected from the population by medical institutions, organizations, or governmental institutions. The leading source of survey data on health in the United States is the National Health Interview Survey (NHIS) \citep{NHIS-original}. The NHIS is an ongoing cross-sectional household study in the United States collecting monthly samples on the general health of the civilian, non-institutionalized population. The design of the survey ensures that each monthly sample is nationally representative. Each year, approximately $50,000$ samples are collected, covering different topics such as health habits and conditions, as well as topics in relation to socioeconomic factors, including education and income. Among the covered topics in health is chronic pain, in which several categories of pain are represented. As such, the NHIS rightfully constitutes the basis for a lot of research related to chronic pain. An issue in using this data, however, is caused by the complexity introduced by survey design. Since the survey design of the NHIS is multistaged and makes use of geographically clustered sampling techniques, the estimates derived from models using their data may arrive at poor conclusions. In addition to the design, the NHIS has, in several years, oversampled some subpopulations (e.g., Black, Hispanic, Asian) to better assess the health of these groups. To produce correct statistical estimates, the NHIS provides sampling weights, which are used in statistical procedures to account for the design of the survey. Nonetheless, a lot of analyses neglect the potential importance of the survey design, though it should, in principle, always be accounted for.

%Age period cohort models
A prominent model in temporal analyses of health-related data is the age-period-cohort (APC) model. APC models have long been useful tools in the temporal analyses of health-related data, allowing for assessment of temporal trends of mortality, prevalence, and incidence rates by examining the data on three temporal scales \citep{APC-OLD, APC-OLD-2, APC-OLD-3}. Although the APC model was initially proposed in a frequentist framework using maximum likelihood methods for inference, Bayesian methods have become more frequent, incorporating a smoothing effect for each of the age, period and cohort effects \citep{APC-Bayesian-Yang, APC-Bayesian-Held}. Of particular interest in the Bayesian framework is the multivariate APC (MAPC) model \citep{hansell2003copd, jacobsen2004women}, which allows for analysis of heterogeneous time trends over different strata, such as gender, wealth, geographical area, or in the context of this thesis, education. By sharing some effects between the different strata, linear combinations of the effects across strata become identifiable, in turn mitigating the infamous problem of non-identifiability of linear time trends in APC models, while still facilitating for comparative inference \citep{APC-Bayesian-Andrea}. 

%Thesis outline
In this thesis, the application of the MAPC model as a Bayesian hierarchical model is investigated using NHIS data on chronic back pain stratified by education in the adult US population aged 25-84 in the period of 1997-2018. The analysis will be carried out separately for males and females, due to potential heterogeneity in the evolution of the educational groups over time. The prior is a vital component of any Bayesian hierarchical model, therefore, a prior is elicited based on expert knowledge (EK) provided from a sociologist (Anna Zajacova, University of Western Ontario), and is based on intuitive and principled approaches to prior elicitation \citep{PC-priors,Jointprior}. In addition, to investigate the potential impact of the complex multistaged and geographically clustered survey design of the NHIS, a recently-proposed two-step method for incorporating survey design weights in the Bayesian MAPC, suggested by \cite{SurveyDesignMercer}, will be used. In Section \ref{section:surveyData}, survey data obtained from the NHIS is presented and discussed in detail along with an exploratory analysis. Further details on the complex survey design of the NHIS are also provided and discussed. Section \ref{section:Age-period-cohort-models} then introduces the APC model in its original univariate formulation, followed by its extension to facilitate stratification in the MAPC model. Methods for cross strata inference in these models are also presented and discussed, which aims to aid interpretation of the upcoming analysis of the fitted models. After which, Section \ref{section:BayesianInference} elaborates on latent field prior and hyperprior assignment in the Bayesian variant of the MAPC model, along with a discussion of prior frameworks to be used in the application. The application of the models on the NHIS data readily follows in Section \ref{section:Application1}, in which the implementation of the models and the prior elicitation is described in detail. The models are implemented in the \texttt{R} programming language, using the packages \texttt{INLA} (available at \href{www.r-inla.org}{www.r-inla.org}) for approximate Bayesian inference, and \texttt{makemyprior} \citep{MMPPackage, MMP} for prior construction. Model selection using logarithmic score \citep{proper-scoring} is also carried out to carefully select the best combination of education-specific model components before presenting, discussing and interpreting the results. The complex survey design is then accounted for in our model in Section \ref{section:application2}, using the \texttt{survey} package for survey design corrected estimates, and following the approach for incorporation in the Bayesian MAPC models presented in \cite{SurveyDesignMercer}. Based on the results and consultation from the aforementioned sociologist, additional analyses to further investigate the back pain trends for participants with LHS/GED diploma is conducted. Finally, closing remarks on the project as a whole follows in Section \ref{section:application2:conclusion}.

%UN development goals stuff
In relation to the Sustainable Development Goals \citep{UNSDG} presented by the United Nations, the work done in this thesis contributes to goals 3 and 10. Goal 3, \textit{Ensure healthy lives and promote well-being for all at all ages} is reflected in the efforts to investigate the temporal intricacies of the observed health disparities in LHS/GED diploma holders, a crucial step in ensuring health for all. By shedding light on the matter, it advocates for measures to be taken to ensure that the well-being of the less-educated population is on par with that of the higher-educated population. Goal 10, \textit{Reduce inequality within and among countries}, is also reflected in this work for the same reasons.

The code producing all models and visualizations of the results in all upcoming sections is (soon, private for now) publicly available on GitHub at \url{https://github.com/Markutr/Masteroppgave}.

%Disclaimers
Though in the field of Bayesian disease mapping, this thesis builds upon my previous project work \citep{Prosjektoppgave} done in anticipation of this thesis. As a consequence, some sections from the project regarding Bayesian inference and prior elicitation have been repurposed for the context of this thesis, and these sections will therefore bear striking resemblance to the original material. Specifically, the contents of Sections \ref{section:component_specific_priors}, \ref{section:joint_priors}, \ref{section:INLA} and Appendix \ref{section:disease-mapping:criteria}, have been repurposed from the project. The project report is available in a different GitHub repository at \url{https://github.com/Markutr/ProjectThesis}.





%The application setting
%An interesting application of these models that will be considered in this thesis is analyzing health trends stratified by education level. One of the strongest correlating factors with adult health in the US is the attainment of education, where higher levels of education have been shown to correlate with better health outcomes \citep{Edcucation-Attainment}. Several studies have already shown evidence of this (refs). Though, disparities with this correlation has also been found, such as for recipients of the general educational development (GED) diploma, whom are qualified as educationally equivalent to high school diploma holders yet suffer worse health outcomes \citep{GED-disparity}. 

%We wish to see if this disparity becomes greater with age. 
%While the health disparity between higher and lower educated individuals has already been established, there is still aspects of this disparity that needs to be researched. One such aspect is the temporal evolution of health outcomes, where the effect of education in health outcomes may become stronger with age, or whether this may be due to a cohort or period effect. 

%We wish to see if we use more features at an individual spectrum. 
%Investigate further disparities at individual level?

%Papers analyzing NHIS data from this period: \citep{GED-disparity}\citep{Zajacova2021-ee} \citep{ZAJACOVA20201270} \citep{sleep}

%We wish to do something about the complex survey design
%These data also present an interesting opportunity to develop methods to account for the complex survey design of the data. 
