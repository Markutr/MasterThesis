\section*{Sammendrag}
Kroniske ryggsmerter er et globalt utbredt helseproblem som påvirker den generelle og psykologiske helsen til de berørte individene, med økonomiske konsekvenser på både personlig og samfunnsmessig nivå. Forskning har funnet at ryggsmerter er mer fremtredende blant voksne med lavere oppnådd utdanning, og forskjellene mellom utdanningsgrupper mistenkes å være større i nyere fødselskohorter.

I denne oppgaven brukes Bayesianske multivariate alder-periode-kohort-modeller til å analysere tidsmessige trender i ryggsmerter over alder, perioder og fødselskohorter hos voksne i USA, stratifisert etter nivå på oppnådd utdanning. Dataene, hentet fra National Health Interview Survey i perioden 1997 til 2018, omhandler voksne i alderen 25 til 84 år, kategorisert i fire utdanningsgrupper. Utjevningspriorifordelinger brukes til å stabilisere estimatene av tidstrendene, med hyperpriorifordelinger dannet ved hjelp av ekspertkunnskap supplert av en sosiolog. Det komplekse undersøkelsesdesignet er gjort rede for ved hjelp av en nylig foreslått to-trinns metode. For fullstendighetens skyld inkluderer analysen vår også modellvalgprosedyrer, sensitivitetsanalyse av priorifordelingene og videre undersøkelser av underpopulasjoner.

Vi fant sterke forskjeller i ryggsmertetrender over alder og fødselskohorter på alle nivåer av oppnådd utdanning sammenlignet med nivået "bachelor or above". Spesielt store forskjeller ble identifisert for utdanningsgruppene "less than high school" og "general educational development". Trendene over alder ser ut til å øke mellom 25 og 60 år, og avtar deretter. I tillegg ble det bekreftet at ulikhetene i ryggsmerte også økte med nyere kohorter. I det hele viste det seg at metoden som ble brukt for å redegjøre for undersøkelsesdesignet var effektiv, og den anbefales dermed i fremtidige anvendelser av alder-periode-kohort-modeller.
