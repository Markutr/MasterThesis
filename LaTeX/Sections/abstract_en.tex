\section*{Abstract}
Chronic back pain is a globally prevalent health problem, compromising the general and psychological health of affected individuals, alongside economic consequences on both personal and societal levels. Back pain is found to be more prominent among adults with lower educational attainment, and the disparities between educational groups are suspected to be wider in more recent birth cohorts.

In this thesis, Bayesian multivariate age-period-cohort models are used to analyse temporal trends along age, period, and birth cohorts in chronic back pain in United States adults, stratified by the level of attained education. The data, taken from the National Health Interview Survey in the period 1997 to 2018, focuses on adults aged 25 to 84, and they are categorized into four educational groups. Smoothing priors are used to stabilize the estimates of the time trends, with joint hyperpriors elicited using expert knowledge provided by a sociologist. The complex survey design is accounted for using a recently-proposed two-step approach. For completeness, our analysis also includes model selection procedures, prior sensitivity analysis, and further investigations of subpopulations.

We found strong disparities in back pain trends over age and birth cohorts in all levels of attained education compared to the level "bachelor or above". Particularly strong disparities were identified for the "less than high school" and "general educational development" educational groups. The trends over age increased noticeably between ages 25 and 60, and then decreased afterwards. Moreover, the back pain disparities were also confirmed to have a distinct increase with more recent cohorts. Overall, the approach used to account for the survey design proved effective, and is recommended in future applications of age-period-cohort models.

